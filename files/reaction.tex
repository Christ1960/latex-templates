\documentclass{book}

\usepackage[round]{natbib}
\usepackage{blindtext}
\usepackage{tipa}
\usepackage{cgloss4e}
\usepackage{gb4e}
\usepackage{qtree}
\usepackage{enumerate}
\usepackage{longtable}

\title{The Red-pilling}
\author{Luke Smith}

\begin{document}

\maketitle


\chapter{Evil Revelations}

I always thought that I was red-pilled. As I said, I always dissented, mostly quietly, the socially acceptable narrative on numerous issues, perhaps most prominently race. This was an inevitable side-effect of my upbringing and maturation in the blacker places in Atlanta. Especially then, once I had shed my earlier flirtation with socialism and Keynesian pseudoscience, I thought that I had reached the summit. Reading Rothbard had allowed me to finally unify my disparate skepticisms of consensus opinion, although I never explicitly identified as a libertarian. I finally find the ``right'' opinions. Logically unassailable yet still in a way very consistent and intuitive, and still undogmatic enough for elaboration or further inquiry.

In a way, I was right in that. I look back at Rothbard and struggle to fault him or the other radical Right libertarians. I disagree on philosophical grounds, particularly the philosophy of science and praxeology, and as to the existence of effective demand and other emergent economic properties, but still libertarians seem to be ``right'' about everything.

But being right about everything isn't Enlightenment. Nor is it being red-pilled. I learned this soon enough.

Eventually I was directed to Radishmag (http://radishmag.wordpress.com). Now unupdated, Radishmag was a longwinded Wordpress site which instead of publishing traditional blog posts, worked up longer, book-like entries of tens of thousands of words. The entries were painstakingly researched, filled to the brim with clandestine original sources, and most importantly, anything but dry or boring.

Radishmag was Reactionary. An indentificatory word I'd run across only several times before even on the Internet. Reactionaries are of course maximally Far-Right, the term originally being used to the opponents of the French Revolution. And several years earlier, although I had missed it directly, a new movement of Neo-Reactionaries had been reignited, principally under the sway of a computer programmer Curtis Yarvin (whose net name was Mencius Moldbug). More on his work later.

Radishmag was my first, and perhaps most important exposure to the movement, and as such, it's important to understand Radishmag's style and content, which go hand in hand. Radishmag was deliberately provocative. Its anonymous author glued together his encyclopedic posts with openly hostile ravings against blacks and feminists and journalists (but mostly blacks), his Twitter even more inflammatory.

Something clicked upon seeing this. Most people leave in total fear of being called racist or sexist or bigotted or whathaveyou. I started to realize that much of my kind of argumentation, and the argumentation of others was slavish to this fact. We're raised to impute enormous power to these words, even when unjustly used (which they usually are). But what happens when you\ldots don't? What if you \emph{not only} stop caring about being called names, but what if you embrace and indeed seek out controversy? Words are most powerful when they keep men silent. But what if they don't?

I had the reaction to Radishmag the author no doubt intended. I was initially disgusted, but no so much to no paruse the content. There was no doubt that I'd find something novel, even if wedded with what I originally thought to be misplaced inflammatoriness. 





I don't remember what article it was I was reading. But there finally came a time when I felt a feeling I hadn't felt in a long, long time. It might've been the article on slavery or colonialism, I think slavery. As I stared into the e-reader, scanning the text, sometime late after midnight, I felt that deep tar-black feeling in the chest that the religious feel when they feel like they are staring into the face of pure evil. After years of explicit atheism, I didn't know my mind would ever produce that kind of defensive mechanism ever again. But here I was, for all of my correctness of belief, my soul was afraid of what kind of mortal sin I might be closely studying.

tp

\bibliographystyle{apalike}
\bibliography{$HOME/Documents/latex/uni.bib}

\end{document}
