\documentclass{book}

\usepackage[round]{natbib}
\usepackage{blindtext}
\usepackage{tipa}
\usepackage{cgloss4e}
\usepackage{gb4e}
\usepackage{qtree}
\usepackage{enumerate}
\usepackage{longtable}

\title{A Phenomonology of Linguistics}
\author{Luke Smith}

\begin{document}

\maketitle

\chapter{Introduction}

This book isn't necessarily intended to be read during the 21\textsuperscript{st} century. People of our times \emph{are} highly encouraged to read it, but the deliberate purpose here is to summarize the findings of modern linguistics in a way legible to anyone of any time, independent of theoretical framework.

The thought behind the book has a brief story. During the summer of 2016, I recommened Ryan to read Nassim Taleb's \textit{Antifragile}, a book with incredibly important practical and theoretical statements

The core message of that book was simple. Complex systems can be fragile, meaning that they are harmed by stress or change, robust, meaning that they are impervious to the effects of change or they can be \emph{antifragile}, meaning that they \emph{improve} when exposed to these kinds of disorder. After enjoying the book himself, Ryan brought to my attention a minor note that Taleb had brought up in a small section of the book.

Scientific theories are by nature fragile. A theory is only alive as long as we have failed to unearth contradictory data. Theoretical ``frameworks'' often last longer, but are only more long-lived in that they often take on an amorphous and amiguous state such that they escape falsifiability by technicality.

While theories are fragile, a \emph{phenomenology} is robust or antifragile. By phenomenology, we mean a catalog of empirical facts and the patterns they appear in without any strong reference to a particular theory. Phenomenologies are at \emph{worst} robust, in that the catalog of what we know about language never decreases as theories come and go, but usually, it is \emph{antifragile}: as new theories with new persepectives open us up to new empirical vistas which expand our phenomenology yet further.


\section{The Twilight of Generative Linguistics}

If you're reading this section outside of the 21\textsuperscript{st} century, feel free to skip this section.

There might still be a school of thought called Generative Linguistics in 100 years, but if there is, it no doubt will be in name only. Generative Linguistics has already changed to something entirely distinct from what it had been in the 1950's.

We all know that Generative Syntax ``won'' the ``linguistics wars'' against Generative Semantics, entirely consuming the movement, but then somehow Generative Semantics came bursting out of its stomach.

Linguistics particularly is a field overripe with superfluous theory. Any linguist should understand that frequent trial of searching for empirical facts in articles only ten years old and having to shovel through tons of already obsolete or outdated theoretical baggage before finding the small desired phenomenological nugget. Old articles often have to be ``translated'' into Minimalist verbiage to make sense to new students, while new articles often have to be ``translated'' into classical Government and Binding for older professors who have been out of the loop.

The theory, vocabulary and metaphors are constantly changing, and these changes, such as the shift to Minimalism, are usually motivated not by new empirical truths, but by a theoretical excercise for economy which too often flippantly discarded unworkable phenomenon into linguistics's various theoretical rubbish bins: ``post-syntacic phenomena,'' ``pragmatics,'' ``performance,'' ``phonological form,'' etc.



We're not saying this to put salt in the wound of those many older professors with strong emotional ties to the movement. Nor are we trying to propose our own alternative theory in this book. The important thing is simply that this theoretical framework, like all others, is fragile and moribund. To keep the fire burning, we must 





The voice of Andrew Carnie resounds in our heads. ``You can't say anything important about linguistics without formal theory!''

Andrew is right here, but in an unimportant way. This can be compared the the kind of exagerated pseudo-skepticism currently plaguing the social sciences. 


This kind of tactical nihilism is a gag rule on useful inquiry, and ironically is often a 



Because we can never truly look at something ``objectively'' or ``without theory'' needn't mean that we shouldn't try. Most importantly, once we state what we see, even if our view is biased or laden with tacit theoretical assumptions, we or others can take a step back and look for the patterns underlying our statements and ``deconstruct'' even deeper patterns that we take for granted or assume unwarrentedly.

This is precisely the point. Hyper-theoretical thinking too often blinds us to data. As Noam Chomsky has been wont to say, science is about ``allowing one's self to be puzzled.''



\bibliographystyle{apalike}
\bibliography{$HOME/Documents/latex/uni.bib}

\end{document}
